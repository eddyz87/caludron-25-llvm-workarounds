\batchmode
\PassOptionsToPackage{prologue,dvipsnames}{xcolor}
% don't show red boxes around lexing errors
\AtBeginEnvironment{minted}{\renewcommand{\fcolorbox}[4][]{#4}}
\documentclass{beamer}
\setbeamersize{text margin left=15pt,text margin right=15pt} 
\setbeamertemplate{frametitle continuation}{\usebeamerfont{frametitle}\small{(\insertcontinuationcount)}}

%%\usepackage[dvipsnames]{xcolor}

\usepackage{minted}
\setminted{escapeinside=!!}
\setminted{fontsize=\small}
%%\setminted{ignorelexererrors=true}

\usepackage{parskip}
\setlength{\parskip}{\medskipamount} 

%%\usepackage{noto}
\usepackage[scaled]{beramono}
\usepackage[T1]{fontenc}

\newcommand{\code}[1]{\texttt{#1}}
\newcommand{\marker}[2]{\colorbox{#1}{#2}}
\newcommand{\green }[1]{\marker{SpringGreen!40}{#1}}
\newcommand{\yellow}[1]{\marker{Goldenrod!40}{#1}}
\newcommand{\blue  }[1]{\marker{SkyBlue!40}{#1}}

\title{LLVM BPF backend optimizer workarounds}
\date{}

\begin{document}

\frame{\titlepage}

\begin{frame}[fragile]
\frametitle{Preface}

\begin{itemize}
\item Optimizing compiler transforms the structure of the input C code.
\item Sometimes such transformations deter Linux kernel BPF verifier
      from reasoning about the program.
\item LLVM BPF backend accumulated a number of transformations
      aimed at preventing certain optimizations to happen.
\item This presentation to enumerates these backend transformations.
\item The goal of the presentation is to discuss how GCC BPF backend
      addresses or plans to address the same issues.
\item Another goal is to resurrect discussion about \code{-Overifyable}.
\end{itemize}

%% Over the years, LLVM BPF backend accumulated a number of
%% transformations aimed at preserving the structure of the input C code
%% from optimizer rewrites, where those rewrites conflict with kernel BPF
%% verifier understanding the code.

%% I'll give a short overview of the BPF verifier itself, and describe
%% the patterns each such transformation tries to preserve.

\end{frame}

\begin{frame}[allowframebreaks,fragile]
  \frametitle{Kernel BPF verifier}

  At it's core Linux Kernel BPF verifier is a path-tracing machine.
  \begin{columns}[t]
    \column{0.45\textwidth}
    \scriptsize
    \begin{itemize}
    \item It simulates instruction execution one-by-one.
    \item Forks the state at branching points.
    \item Tracks register and stack state.
    \end{itemize}
    \column{0.65\textwidth}
    \begin{minted}[highlightlines={2,10,13,22},fontsize=\tiny]{c}
Live regs before insn:
    0: .......... (85) call bpf_get_prandom_u32
    1: 0......... (bf) r1 = r0
    2: 01........ (07) r1 += 5
.-- 3: 01........ (35) if r0 >= 0x5 goto pc+5
|   4: .1........ (18) r0 = 0xffffc90000188000
|   6: 01........ (0f) r0 += r1
|   7: 0......... (71) r0 = *(u8 *)(r0 +0)
|   8: 0......... (95) exit
'-> 9: .......... (b7) r0 = 42
   10: 0......... (95) exit

0: (85) call bpf_get_prandom_u32 ; !\scalebox{.9}[1.0]{R0=scalar()}!
1: (bf) r1 = r0                  ; !\scalebox{.9}[1.0]{R0=scalar(id=1) R1=scalar(id=1)}!
2: (07) r1 += 5                  ; !\scalebox{.9}[1.0]{R1=scalar(id=1+5)}!
3: (35) if r0 >= 0x5 goto pc+5   ; !\scalebox{.9}[1.0]{R0=scalar(id=1,umin=0,umax=4)}!
4: (18) r0 = 0xffffc90000188000  ; !\scalebox{.9}[1.0]{R0=map\_value()}!
6: (0f) r0 += r1                 ; !\scalebox{.9}[1.0]{R0=map\_value(umin=5,umax=9)}!
7: (71) r0 = *(u8 *)(r0 +0)      ; !\scalebox{.9}[1.0]{R0=scalar(umin=0,umax=255)}!
8: (95) exit

from 3 to 9: !\scalebox{.9}[1.0]{R0=scalar(id=1,umin=5) R1=scalar(id=1) R10=fp0}!
9: (b7) r0 = 42                  ; !\scalebox{.9}[1.0]{R0=42}!
10: (95) exit      
    \end{minted}
  \end{columns}

  \framebreak

    \scriptsize
    \begin{itemize}
    \item Interval and known bits abstract domains are used to represent values.
    \item Linear relations of form $r_u + a == r_v + b$ are tracked
      for 64-bit registers and slots.
    \item Only addition and jump instructions preserve the relations.
    \end{itemize}

    \begin{minted}[fontsize=\tiny,highlightlines={2,3,4,6}]{c}
0: (85) call bpf_get_prandom_u32    ; R0=scalar()
1: (bf) r1 = r0                     ; R0=scalar(id=1) R1=scalar(id=1)
2: (07) r1 += 5                     ; R1=scalar(id=1+5)
3: (35) if r0 >= 0x5 goto pc+5      ; R0=scalar(id=1,umin=0,umax=4)
4: (18) r0 = 0xffffc90000188000     ; R0=map_value()
6: (0f) r0 += r1                    ; R0=map_value(umin=5,umax=9)
7: (71) r0 = *(u8 *)(r0 +0)         ; R0=scalar(umin=0,umax=255)
8: (95) exit

from 3 to 9: R0=scalar(id=1,umin=5) R1=scalar(id=1) R10=fp0
9: R0=scalar(id=1,umin=5) R1=scalar(id=1) R10=fp0
9: (b7) r0 = 42                     ; R0=42
10: (95) exit      
    \end{minted}

    
%     \begin{minted}[fontsize=\tiny,highlightlines={2,3,4,6}]{c}
% 0: (85) call bpf_get_prandom_u32    ; R0=scalar()
% 1: (bf) r1 = r0                     ; R0=scalar(id=1) R1=scalar(id=1)
% 2: (07) r1 += 5                     ; R1=scalar(id=1+5)
% 3: (35) if r0 >= 0x5 goto pc+5      ; R0=scalar(id=1,umin=0,umax=4)
% 4: (18) r0 = 0xffffc90000188000     ; R0=map_value()
% 6: (0f) r0 += r1                    ; R0=map_value(umin=5,umax32=9)
% 7: (71) r0 = *(u8 *)(r0 +0)         ; R0=scalar(umin=0,umax=255)
% 8: (95) exit
%     \end{minted}

\end{frame}

\begin{frame}[allowframebreaks,fragile]
  \frametitle{Address calculation hoisting}

  From C point of view the two programs below are identical:
  \vspace{0.5cm}

  \begin{columns}
    \column{0.35\textwidth}
    \begin{minipage}{\textwidth}
      \begin{minted}{C}
off = ...
if (off < 42) {
  ptr = packet + off;
  ... *ptr ...
}
      \end{minted}
    \end{minipage}
    \column{0.1\textwidth}
    $\rightarrow$
    \column{0.5\textwidth}
    \begin{minipage}{\textwidth}
      \begin{minted}{C}
off = ...
ptr = packet + off;
if (off < 42) {
  ... *ptr ...
}
      \end{minted}
    \end{minipage}
  \end{columns}

  \framebreak

Verifier does not track relations between scalars and pointers, so it
can't propagate range backwards from \code{off} to \code{ptr}.

  \begin{minted}{C}
off = ...
ptr = packet + off;
if (off < 42) {
  ... *ptr ... /* !\textcolor{red}{verification error here}! */
}
  \end{minted}

  \framebreak

LLVM BPF backend attempts to block such code motion by inserting
special intrinsic calls\footnote{\tiny{Handled by \code{BPFAdjustOptImpl::avoidSpeculation()}.}}:

  \begin{minted}{C}
!\blue{off}! = ...
if (!\yellow{off < 42}!) {
  off = !\textcolor{Red}{@llvm.bpf.passthrough.i32.i32}!(<uid>, i32 off);
  ptr = packet !\green{+ off}!;
  ... *ptr ...
}
  \end{minted}

By matching value \blue{definition}, \yellow{icmp} and \green{address calculation} pattern.

\end{frame}

\begin{frame}[allowframebreaks,fragile]
  \frametitle{\code{x < a || x > b}}

  LLVM's \code{InstCombine} is a huge set of peephole rewrite patterns. \\
  One of these patterns looks as follows:

  \vspace{0.5cm}

  \begin{columns}
    \column{0.3\textwidth}
    \begin{minipage}{\textwidth}
      \begin{minted}{C}
if (v < 1)
        return 0;
if (v > 7)
        return 0;
return arr[v];
      \end{minted}
    \end{minipage}
    \column{0.1\textwidth}
    $\rightarrow$
    \column{0.45\textwidth}
    \begin{minipage}{\textwidth}
      \begin{minted}{C}
t = v;
t -= 8;
if ((u32)t < (u32)-7)
	return 0;
return arr[v];
      \end{minted}
    \end{minipage}
  \end{columns}

  \framebreak

  This is a problem for verifier:

  \begin{minted}[fontsize=\fontsize{7}{8},frame=lines,framesep=5mm,highlightlines=3]{c}
  ...  w1 = v, w2 = t ...
  3: (bc) w2 = w1                       ; R1=R2=scalar(id=1,umin=0,umax=0xffffffff)
  4: (04) w2 += -8                      ; R2=scalar(umin=0,umax=0xffffffff...)
  5: (a6) if w2 < 0xfffffff9 goto pc+5  ; R2=scalar(...smin32=-7,smax32=-1...)
  6: (bc) w1 = w1                       ; R1=scalar(id=1,umin=0,umax=0xffffffff...)
  7: (18) r2 = <arr> ll                 ; R2=map_value(...)
  9: (0f) r2 += r1
  10: (71) r0 = *(u8 *)(r2 +0)
  R2 unbounded memory access, make sure to bounds check any such access
  \end{minted}

    %%\vspace{0.5cm}

At (4) verifier drops relationship between `r1` and `r2`, as it does
not track linear relations after 32-bit operations.

  \framebreak

The workaround is two hide the fact that both conditionals work on a
same value from optimizer.

Match a sequence of basic blocks:

\begin{minted}[fontsize=\small]{llvm}
entry:
  %cmp = icmp ult i64 %x, 1
  br i1 %cmp, label %if.then, label %if.end

if.end:
  %cmp1 = icmp ugt i64 %x, 7
  br i1 %cmp1, label %if.then2, label %if.end3
\end{minted}

  \framebreak

  And break the pattern for \code{InstCombine} by obfuscating
  \code{\%x} value for the first \code{icmp}
  \footnote{\tiny{\parbox{\textwidth}{
        \code{Handled by BPFAdjustOptImpl::serializeICMPCrossBB()} and\\
        \code{BPFAdjustOptImpl::serializeICMPInBB()}.  }}}:

\begin{minted}[fontsize=\small,highlightlines=3]{llvm}
entry:
  %cmp = icmp ult i64 %x, 1
  %2 = call i1 @llvm.bpf.passthrough.i1.i1(i32 2, i1 %cmp)
  br i1 %2, label %if.then, label %if.end

if.end:
  %cmp1 = icmp ugt i64 %x, 7
  br i1 %cmp1, label %if.then2, label %if.end3
\end{minted}
    
\end{frame}

\begin{frame}[allowframebreaks,fragile]
  \frametitle{\code{if (x <= $2^n$) \hspace{1em}...}}

Same \code{InstCombine} pass applies the following transformation:

  \vspace{0.5cm}

  \begin{columns}
    \column{0.45\textwidth}
    \begin{minipage}{\textwidth}
      \begin{minted}{C}
int foo(long x) {
  if ((unsigned)x <= !$2^n$!)
    return arr[x];
  return 0;
}
      \end{minted}
    \end{minipage}
    \column{0.1\textwidth}
    $\rightarrow$
    \column{0.45\textwidth}
    \begin{minipage}{\textwidth}
      \begin{minted}{C}
int foo(long x) {
  if (x & (!$2^{32} - 2^n$!))
    return arr[x];
  return 0;
}
      \end{minted}
    \end{minipage}
  \end{columns}

  \framebreak
  
  This is a problem for verifier:

  \begin{minted}[fontsize=\fontsize{7}{8},frame=lines,framesep=5mm,highlightlines={4,5}]{c}
 6: (18) r2 = 0xfffffffe              ; R2=0xfffffffe
; if ((unsigned)x <= 1) {
 8: (bf) r3 = r1                      ; R1=scalar(id=1) R3=scalar(id=1)
 9: (5f) r3 &= r2                     ; R3=scalar(...umin=0,umax=0xfffffffe...)
10: (55) if r3 != 0x0 goto pc+6       ; R3=0
; return arr[x];
11: (18) r2 = <arr> ll                ; R2=map_value(...)
13: (0f) r2 += r1
math between map_value pointer and register with unbounded min value is not allowed
  \end{minted}

At (9) \code{r3 \&= r2} breaks the relation between \code{r3} and \code{r2}.\\
At (10) no range for \code{r1} inferred from \code{r3 == 0} branch entry.

  \framebreak

  The workaround is two hide comparison from optimizer, by mathcing
  this specifi pattern:

\begin{minted}[fontsize=\small]{llvm}
  %cmp = icmp ule i32 %conv, 1   ; <------- !$2^n$! RHS
  br i1 %cmp, label %if.then, label %if.end
\end{minted}

And replace it with the a call to a backend specific intrinsic:

\begin{minted}[fontsize=\small]{llvm}
  %0 = call i1 @llvm.bpf.compare.i32.i32
                  (i32 37, i32 %conv, i32 1)
  br i1 %0, label %if.then, label %if.end
\end{minted}
  
\end{frame}

\begin{frame}[allowframebreaks,fragile]
  \frametitle{\large\code{\_\_attribute\_\_((preserve\_static\_offset))}}

  Verifier has a concept of ``context'' types:
  \begin{itemize}
  \item describe input parameters of BPF programs
  \item C-level representation for BPF programs differs from physical layout
  \item as a part of the verification process, load and store
    instructions with static offsets are patched to use correct physical layout.

  For example:

  \begin{minted}[fontsize=\footnotesize]{c}
int foo(struct bpf_sockopt *ctx) {
	... ctx->retval ...      
}
  \end{minted}

  Is translated as \code{*(u32 *)(r1 + 36)}, and is patched to an
  equivalent of:

  \begin{minted}[fontsize=\footnotesize]{c}
(struct sockopt_kern *)r1->current_task->bpf_ctx->retval
  \end{minted}
  \end{itemize}

  \framebreak

  Such mechanics fails if compiler decouples computation of the
  access address and load/store instruction, e.g.:

  \begin{columns}
    \column{0.35\textwidth}
    \begin{minted}[fontsize=\footnotesize]{c}
switch (ctx->level) {
case 10:
  g = !\textcolor{Red}{bar}!(ctx->sk->family);
  break;
case 20:
  g = !\textcolor{Red}{bar}!(ctx->optlen);
  break;
}
return g % 2;
    \end{minted}

    \column{0.05\textwidth}
    $\rightarrow$
    \column{0.45\textwidth}

    \begin{minted}[fontsize=\footnotesize]{c}
switch (ctx->level) {
case 10:
  p = (void *)ctx->sk + 4;
  break;
case 20:
  p = ctx + 32;
  break;
}
g = !\textcolor{Red}{bar}!(*p);
return g % 2;
    \end{minted}
  \end{columns}

  \framebreak

  Verifier rejects programs after such transformations:

  \begin{minted}[fontsize=\fontsize{7}{8},frame=lines,framesep=5mm,highlightlines={7,8}]{c}
; switch (ctx->level) {
0: (61) r2 = *(u32 *)(r1 +24)         ; R1=ctx() R2=scalar(...)
1: (16) if w2 == 0x14 goto pc+5       ; R2=scalar(...)
2: (b4) w0 = 0                        ; R0=0
3: (56) if w2 != 0xa goto pc+7        ; R2=10
; g = bar(ctx->sk->family);
4: (79) r1 = *(u64 *)(r1 +0)          ; R1=sock()
5: (07) r1 += 4
R1 pointer arithmetic on sock prohibited
  \end{minted}

  \framebreak

  The workaround is to use a special attribute that informs BPF
  backend to apply the following transformation before main optimizer
  passes:

  \begin{minted}[fontsize=\footnotesize]{c}
struct bpf_sockopt { ... } __attribute__((preserve_static_offset));
...
switch (!\textcolor{Red}{@llvm.bpf.getelementptr.and.load.i32}!(ctx, 24)) {
case 10:
  g = bar(ctx->sk->family); // offsets of 0 are not tinkered with
  break;
case 20:
  g = bar(!\textcolor{Red}{@llvm.bpf.getelementptr.and.load.i32}!(ctx, 32));
  break;
}
  \end{minted}

  \code{llvm.bpf.getelementptr.and.load} calls are replaced by actual
  \code{getelementptr} and \code{load} pairs right before IR
  translation to machine code.

\end{frame}

\begin{frame}
\frametitle{User visible workarounds}

The set of transformations applied by LLVM backend is
nowhere from being a complete remedy. This is evident
from kernel bpf selftests themselves:

\begin{itemize}
\item \code{volatile} is used 302 times in 918 files.
\item \code{asm volatile("" : "+r"(var))} is used 26 times.
\end{itemize}

The issues are mostly similar to those already described. \\
(No detailed case study available).

\end{frame}

\begin{frame}
\frametitle{Conclusion}
\begin{itemize}
\item LLVM BPF backend strategy is so far reactive,
      targeted counter-transformations are added when new
      patterns are identified.
\item What is the strategy for GCC BPF backend?
\item Do we want to agree on certain \code{-Overifyable},
      suggested some time ago by Jose Marchesi? \\
      If we do, how should it be implemented:
      \begin{itemize}
      \item By targeted IR alternations, as LLVM does?
      \item By backend specific checks in optimizations
            themselves \\ (an option rejected by LLVM upstream)?
      \item Given a big number of user-visible workarounds,
            even an optimization pipeline completely separate
            from existing \code{-O2} might be interesting.
      \end{itemize}
\end{itemize}
\end{frame}

\end{document}
